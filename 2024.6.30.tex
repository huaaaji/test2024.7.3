\documentclass{article}
\usepackage[UTF8]{ctex}
\usepackage{multirow}
\usepackage{color}
\usepackage{amsmath}
\begin{document}
	\title{LaTex学习笔记}
	\maketitle
	\section{文档结构}
	\subsection{基本文档结构}
	\textbackslash documentclass[a4paper, 12pt]\{article\}
	\\
\label{ref1}
	\\
	\textbackslash begin\{document\}\\
	  在这里输入文本内容\\
	\textbackslash end\{document\}\\
	
	*注意,反斜杠等保留字在latex输出应注意转义,反斜杠->\textbackslash textbackslash
	而其他的符号等若需转义,则以反斜杠加于之前
	
	\subsection{文档标题}
	\textbackslash maketitle 命令可以给文档创建标题。你需要指定文档的标题。
	如果没有指定日期,就会使用现在的时间,作者是可选的。\\
	以下是示例:\\
	\textbackslash title\{My First Document\}\\
	\textbackslash author\{My Name\}\\
	\textbackslash date\{\textbackslash today\}\\
	\textbackslash maketitle\\
	 注意,标题、作者等内容应于文本内容区域\\
\textbackslash today 是插入当前时间的命令。你也可以输入一个不同的时间,比如 \textbackslash date\{November 2013\}\\
article 文档的正文会紧跟着标题之后在同一页上排版。report 会将标题置为单独的一页。\\
\\
章(Chatpers)、节(Sections)和小节(Subsections)。下列分节命令适用于 article 类型的文档:\\
\textbackslash section\{...\}\\
\textbackslash subsection\{...\}\\
\textbackslash subsubsection\{...\}\\
\textbackslash paragraph\{...\}\\
\textbackslash subparagraph\{...\}\\
\subsection{创建标签}
使用 \textbackslash label\{labelname\} 对章节创建标签。然后输入 \textbackslash ref\{labelname\} 或者 \textbackslash pageref\{labelname\} 来引用对应的章节。\\

下面是一个示范,引用了一个到1.1 基本文档结构的标签\\
文本中(甚至行中)的\textbackslash ref\{ref1\}乃:\ref{ref1}  \\
文本中(甚至行中)的\textbackslash pageref\{ref1\}乃:\pageref{ref1}
\subsection{生成目录}
使用 \textbackslash tableofcontents 在文档中创建目录。通常我们会在标题的后面建立目录。\\
\textbackslash newpage 命令会另起一个页面
\section{文字处理}
\subsection{中文支持}
使用 CTeX 宏包。只需要在文档的前导命令部分添加:\\
\textbackslash usepackage[UTF8]\{ctex\}\\	
且排版时应使用xelatex,因其支持中文字体	

\subsection{字体效果}
\subsubsection{字之大小}
\textbackslash tiny、
\textbackslash scriptsize、
\textbackslash footnotesize、
\textbackslash small、
\textbackslash normalsize、
\textbackslash large、
\textbackslash Large、
\textbackslash LARGE、
\textbackslash huge、
\textbackslash Huge\\
字号一般由特殊的名字定义,它们的大小并不绝对,而是文档默认字体大小的相对值(在\textbackslash documentclass声明中定义)。
此类字体大小之用法,需要将改变大小的文本与命令置于一个花括号中\\
{如In this example the \{\textbackslash huge huge font size\} is set即In this example the {\huge huge font size} is set}
\subsubsection{字体样式之使用}
\label{tabular1}
\begin{tabular}{c|c|c|c|c|}\centering
序号&样式&命令&转换命令&示例\\\hline
1&medium 中等		&\textbackslash textmd\{text\}					&\textbackslash textbackslash mdseries	&\textmd{text 0123 测试}\\\hline
2&bold 粗体			&\textbackslash textbf\{text\}					&\textbackslash bfseries					&\textbf{text 0123 测试}\\\hline
3&upright 直立的		&\textbackslash textup\{text\}					&\textbackslash upshape					&\textup{text 0123 测试}\\\hline
4&italic 斜体		&\textbackslash textit\{text\}					&\textbackslash itshape					&\textit{text 0123 测试}\\\hline
5&slanted 斜体		&\textbackslash textsl\{text\}					&\textbackslash slshape					&\textsl{text 0123 测试}\\\hline
6&small caps&\multirow{2}{*}{\textbackslash textsc\{text\}}&\multirow{2}{*}{\textbackslash scshape}&\multirow{2}{*}{\textsc{text 0123 测试}}\\
7&小型大写字母& & &\\\hline
8&underlined words	&\textbackslash underline\{text\}				& 												&\underline{text 0123 测试}\\\hline
9&teletype			&\textbackslash texttt\{text\}					&												&\texttt{text 0123 测试}\\\hline
10&						&\textbackslash textrm\{text\}					&												&\textrm{text 0123 测试}\\\hline
11&						&\textbackslash textsf\{text\}					&												&\textsf{text 0123 测试}
\end{tabular}
\subsection{彩色字体}
需要使用包(package)。你可以引用很多包来增强 LaTeX 的排版效果。包引用的命令放置在文档的前导命令的位置(即放在 \textbackslash begin\{document\} 命令之前)。使用 \textbackslash usepackage[options\{package\} 来引用包。其中 package 是包的名称,而 options 是指定包的特征的一些参数。\\
使用彩色字体的代码为:\\\{\textbackslash color\{colorname\}text\}\\
例以:
{\color{yellow}yellow}{\color{red}red}{\color{blue}blue}{\color{white}white}\\
同样可以使用 Color 包中的 \textbackslash colorbox 命令来达到。用法如下:\\
\textbackslash colorbox\{colorname\}\{text\},例以:\\
\colorbox{yellow}{yellow} \colorbox{blue}{blue}
\subsection{段落缩进}
LaTeX 默认每个章节第一段首行顶格,之后的段落首行缩进。如果想要段落顶格,在要顶格的段落前加 \textbackslash noindent 命令即可。如果希望全局所有段落都顶格,在文档的某一位置使用 \textbackslash setlength\{\textbackslash parindent\}\{0pt\} 命令,之后的所有段落都会顶格。

\subsection{列表}
LaTeX 支持两种类型的列表:有序列表(enumerate)和无序列表(itemize)。列表中的元素定义为 \textbackslash item。列表可以有子列表。
例以如下代码:
\textbackslash begin\{enumerate\}\\
  \textbackslash item First thing\\
\\
\textbackslash item Second thing\\
    \textbackslash begin\{itemize\}\\
      \textbackslash item A sub-thing\\
\\
      \textbackslash item Another sub-thing\\
    \textbackslash end\{itemize\}\\
\\
  \textbackslash item Third thing\\
\textbackslash end\{enumerate\}\\
效果如下:\\
\begin{enumerate}
  \item First thing

  \item Second thing
    \begin{itemize}
      \item A sub-thing

      \item Another sub-thing
    \end{itemize}

  \item Third thing
\end{enumerate}

可以使用方括号参数来修改无序列表头的标志。例如,\textbackslash item[\-] 会使用一个杠作为标志,你甚至可以使用一个单词,比如 \textbackslash item[One]。

\subsection{注释和空白}
\subsubsection{注释}
使用 \% 创建一个单行注释,在这个字符之后的该行上的内容都会被忽略,直到下一行开始
\subsubsection{空白}
多个连续空格在 LaTeX 中被视为一个空格。多个连续空行被视为一个空行。空行的主要功能是开始一个新的段落。通常来说,LaTeX 忽略空行和其他空白字符,两个反斜杠(\textbackslash \textbackslash )可以被用来换行。
如果你想要在你的文档中添加空格,你可以使用 \textbackslash vaspace\{...\} 的命令。这样可以添加竖着的空格,高度可以指定。如 \textbackslash vspace\{12pt\} 会产生一个空格,高度等于 12pt 的文字的高度。
\subsection{特殊字符}
见前述\\
并注意在使用 \^{} 和 \~{} 字符的时侯需要在后面紧跟一对闭合的花括号,否则他们就会被解释为字母的上标,就像 \textbackslash \^{} e 会变成\~e 。

\section{表格}
表格(tabular)命令用于排版表格。LaTeX 默认表格是没有横向和竖向的分割线的——如果你需要,你得手动设定。LaTeX 会根据内容自动设置表格的宽度。此代码可以创一个表格:
\textbackslash begin\{tabular\}\{...\}
省略号会由定义表格的列的代码替换:\\
l 表示一个左对齐的列;\\
r 表示一个右对齐的列;\\
c 表示一个向中对齐的列;\\
| 表示一个列的竖线;\\
例如,\{lll\} 会生成一个三列的表格,并且保存向左对齐,没有显式的竖线;\{|l|l|r|\} 会生成一个三列表格,前两列左对齐,最后一列右对齐,并且相邻两列之间有显式的竖线。\\\\
表格的数据在 \textbackslash begin{tabular} 后输入:\\
\& 用于分割列;\\
\textbackslash \textbackslash  用于换行;\\
\textbackslash hline 表示插入一个贯穿所有列的横着的分割线;\\
\textbackslash cline\{1-2\} 会在第一列和第二列插入一个横着的分割线。\\
\textbackslash multirow\{A\}\{*\}\{单元格内容\}, A表示所跨行数, 第二个括号可以换成单元格大小等参数。并注意跨行功能在包中,需要加载\textbackslash usepackage\{multirow\}:提供跨行命令
\textbackslash multicolumn\{A\}\{|c|\}\{单元格内容\}, A表示所跨列数,\{|c|\}是表示单元格格式,画单元格左右侧的边界线\\
\textbackslash centering:把表居中\\
最后使用 \textbackslash end\{tabular\} 结束表格。\\
关于表格的使用实例,请见\pageref{tabular1}页的\ref{tabular1}
\\

\section{公式}
\subsection{插入公式}
你可以使用一对 \$ 来启用数学模式,这可以用于撰写行内数学公式。例如 \$1+2=3\$ 的生成效果是 $1+2=3$。\\
如果是生成带标号的公式,可以使用 \textbackslash begin\{equation\}...\textbackslash end\{equation\}\\
例以:\\
\textbackslash begin\{equation\}\\
  1+2=3\\
\textbackslash end\{equation\}\\
乃:\\
\begin{equation}
  1+2=3
\end{equation}
使用 \textbackslash begin\{eqnarray\}...\textbackslash end\{eqnarray\} 来撰写一组带标号的公式。\\
例以:\\
\textbackslash begin\{eqnarray\}\\
  a \& = \& b + c \\
  \& = \& y - z
\textbackslash end\{eqnarray\}\\
乃\\
\begin{eqnarray}
  a & = & b + c \\
  & = & y - z
\end{eqnarray}
要撰写不标号的公式就在环境标志的后面添加 * 字符,如 \{equation*\},\{eqnarray*\}\\
\\
可以发现,使用 eqnarray 时,会出现等号周围的空隙过大之类的问题。\\
可以使用 amsmath 宏包中的 align 环境:\\
\textbackslash usepackage\{amsmath\}\\
...\\
\textbackslash begin\{align\}\\
  a \& = b + c \\
    \& = y - z\\
\textbackslash end\{align\}\\
乃:\\
\begin{align}
  a & = b + c \\
    & = y - z
\end{align}
抑或在行间公式中使用 aligned 环境。它们的名字后面加上星号后,公式就不带标号了。

\subsection{插入符号}
一些基础的符号可以直接键入,但大多数特殊符号需要使用命令来显示

\subsubsection{上标和下标}
\begin{itemize}
\item 上标(Powers)使用 \^{} 来表示,比如 \$n\^{}2\$ 生成的效果为 $n^2$。

\item 下标(Indices)使用 \_{} 表示,比如 \$2\_{}a\$ 生成的效果为 $2_a$。

\item 如果上标或下标的内容包含多个字符,请使用花括号包裹起来。比如 \$b\_{}\{a-2\}\$ 的效果为 $b_{a-2}$。
\end{itemize}

\subsubsection{分数}
\begin{itemize}
\item 分数使用 \textbackslash frac\{numerator\}\{denominator\} 命令插入。比如 \$\$\textbackslash frac\{a\}\{3\}\$\$ 的生成效果为$$\frac{a}{3}$$
\item 分数可以嵌套。比如 \$\$\textbackslash frac\{y\}\{\textbackslash frac\{3\}\{x\}+b\}\$\$ 的生成效果为$$\frac{y}{\frac{3}{x}+b}$$
\end{itemize}

\subsubsection{根号}
\begin{itemize}
\item 我们使用 \textbackslash sqrt\{...\} 命令插入根号。省略号的内容由被开根的内容替代。如果需要添加开根的次数,使用方括号括起来即可。
\item 例如 \$\$\textbackslash sqrt\{y\^{}2\}\$\$ 的生成效果为 $$\sqrt{y^2}$$
\item 而 \$\$\textbackslash sqrt[x]\{y\^{}2\}\$\$ 的生成效果为 $$\sqrt[x]{y^2}$$ 
\end{itemize}

\subsubsection{求和与积分}
\begin{itemize}
\item 使用 \textbackslash sum 和 \textbackslash int 来插入求和式与积分式。对于两种符号,上限使用 \^{} 来表示,而下限使用 \_{} 表示。
\item \$\$\textbackslash sum\_{}\{x=1\}\^{}5 y\^{}z\$\$ 的生成效果为$$\sum_{x=1}^5 y^z$$
\item \$\$\textbackslash int\_{}a\^{}b f(x)\$\$ 的生成效果为$$\int_a^b f(x)$$
\end{itemize}

\subsubsection{希腊字母}
我们可以使用反斜杠加希腊字母的名称来表示一个希腊字母。名称的首字母的大小写决定希腊字母的形态。
\begin{itemize}
\item \$\textbackslash alpha\$=$\alpha$
\item \$\textbackslash beta\$=$\beta$
\item \$\textbackslash delta, \textbackslash Delta\$=							$\delta, \Delta$
\item \$\textbackslash pi, \textbackslash Pi\$=									$\pi, \Pi$
\item \$\textbackslash sigma, \textbackslash Sigma\$=							$\sigma, \Sigma$
\item \$\textbackslash phi, \textbackslash Phi, \textbackslash varphi\$=	$\phi, \Phi, \varphi$
\item \$\textbackslash psi, \textbackslash Psi\$=								$\psi, \Psi$
\item \$\textbackslash omega, \textbackslash Omega\$=							$\omega, \Omega$
\end{itemize}
\end{document}